Neste exercicio vamos criar uma calculadora 

1 Anotação: Sobre 'this'

Sempre que eu quiser referenciar alguma chave de um objeto dentro do proprio objeto 
tenho que usar a palavra this.chave porque assim a engine entende que estou falando
do elemento onde estou referenciando a chave 


clearDisplay(){
    this.display.value = '';
 },

 Aqui estou criando o botão de limpar o input atraves do evento que de click abaixo

 
 if(element.classList.contains('button-clear')){
    this.clearDisplay()
};

ou seja estou pegando o botão atraves da classList e estou mandando o metedo que limpar
o valor do input


Ja aqui estou criando metodo deleteOne que recebe o display e o valor interno dele 
e passo ele no metedo slice que siginifica que estou pegando a string dentro do input
passando o -1, ou seja a cada click um caracter é apagado

deleteOne(){
    this.display.value = this.display.value.slice(0, -1) 
},

E aqui estou selecionando o button delete para que ele escute o click 

if(element.classList.contains('button-del')){
    this.deleteOne()
}


Aqui estou criando um metedo atraves da função eval(). Que é uma função que executa qualquer
string inserida nela como codigo java script, essa função é perigosa pois em projetos reais 
deve ser usada com sabedoria e com os devidos condicionais para que um script malicioso nao seja 
executado atraves do seu codigo

Porem aqui é apenas um exercicio, estou aprendendo agora tambem a linguagem, então no 
metodo eu pego valor dentro do input passo por eval atraves deu um try catch para garantir 
que a conta seja valida, simples assim, porem nao coloco mais restrições no input pois ainda nao 
sei fazer-lo por exemplo deixa que so digite numeros nele

realizaConta(){
    let conta = this.display.value;

    try{
       conta = eval(conta)

       if(!conta){
           alert('Conta Inválida')
           return;
       }
         this.display.value = String(conta)

    }catch(e){
       alert('Conta Inválida')
       return
    }

E Aqui é event do button-eq o botão de igual da calculadora


if (element.classList.contains('button-eq')) {
    this.realizaConta()
}

Tambem coloquei a tecla enter para realizaConta(), ou seja quando eu precionar tecla 
o metodo realizaConta vai ser acionado.


precionaEnter(){
    this.display.addEventListener('keyup', (e) =>{
     if(e.keyCode === 13){
         this.realizaConta()
     }
    })

Esse é o metodo utilizando o addEventListener e a keyCode do tecla Enter
