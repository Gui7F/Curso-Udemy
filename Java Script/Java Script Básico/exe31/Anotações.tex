Esse exercicio começo com a seleção dos objetos atraves do querySelector 
Depois criamos uma fuction criLi para cria um elemento li 
Depois criamos uma fuction criaTarefa com parametro tarefa que recebe nele o valor interno
do input esse parametro e passado para os li.innerHTML com valor que esta sendo criado internamente
o button adicionamos um eventeListenner para o click eventeListenner para assim que clicar 
pega o valor de input e passar para fuction criaTarefa(). E na fuction criaTarefa() adicionamos
lista-tarefa.appendChild() para adicionar a tarefa na nossa ul = lista-tarefa



Agora adicionamos um outro eventeListenner no input, evento de 'keypress' que é quando 
apertamos alguma tecla tambem temos os events de 'keydown' e 'keyup' que são de segurar
e outro de precionar e soltar a tecla respectivamente. Continuando passamos no evento a 
fuction criaTarefa com valor do input 

Agora criamos um fuction para limpar o input assim que ele for criada falicita para add
mais tarefas com facilidade

function limpaInput(){
    inputTarefa.value = ''
    inputTarefa.focus()
}

e chamamos ela na fuction criaTarefa, inputTarefa.focus() siginifica que estamos chamando 
o evento html para ser criado pelo js e voltar o focus para inicio do input. 


function criaBotaoApagar(li){
    const botao = document.createElement('button')
    botao.innerText = 'Remover'
    li.innerText += ' '
    botao.setAttribute('class', 'apagar',)
    botao.setAttribute('title', 'Remover esta tarefa',)

    li.appendChild(botao)
}

Agora criamos esse botao assim que uma tarefa é criada, e aprendemos a função setAttribute() que serve 
para adicionar qualque atributo de qualque elemento o primeiro parametro é o atributo e o segundo é o valor
deste atributo

document.addEventListener('click', function (e){
    const element = e.target
    if(element.classList.contains('remover')){
      element.parentElement.remove()
    }
})

agora aqui capturamos o evento de click no botão de apagar atraves do documento direto e aprendemos agora 
a função de parentElement(), ou seja podemos selecionar o pai do elemento que estamos criando que no caso é 
uma 'li' e remove() apenas remove ela na execução do click

function salvaTarefas(){
   const liTarefas = listaTarefa.querySelectorAll('li')
   const listadeTarefas = [];

   for(let tarefa of liTarefas){
     let tarefaTexto = tarefa.innerText;
     tarefaTexto = tarefaTexto.replace('Remover', '').trim();
     listadeTarefas.push(tarefaTexto)

     console.log(listadeTarefas)
   }

   const tarefasJSON = JSON.stringify(listadeTarefas)
   localStorage.setItem('tarefas', tarefasJSON)
}

Agora nesta fuction criamos um array vazio e selecinamos todas a li, depois iteramos sobre as li pegando todas
depois dentro do laço pegamos o texto dentro das li e passamos para variavel tarefaTexto para ai entao fazer o
replace do texto do botao 'Remover' e para tirar o espaço sobrando usamos trim(), depois disso puchamos os texto 
para dentro do array listadeTarefas 

E logo em baixo convertemos as tarefas para um arquivo JSON, para poder salvar essas tarefas no storege do browser
convertemos em uma stringify, depois pegamos a variavel JSON e setamos cada item que passa por ela no localStorege
do navegador, com nome de 'tarefas'

function recuperarJSON (){
    const tarefas = localStorage.getItem('tarefas')
    const listaDeTarefas = JSON.parse(tarefas)
    for(let index of listaDeTarefas){
        criaTarefa(index)
    }
}

Depois aqui criamos uma outra fuction que recuperarJSON, criamos um variavel que acessa o localStorege e pega 
os valores que estao dentro da entidade 'tarefas' com metodo .getItem(). depois em outra variavel pegamos com metodo
JSON.parse() convertemos as tarefas de volta para string, assim nos iteramos novamente sobre essas tarefas 
e passamos a iteração na nossa function de criaTarefa, assim sempre ficam salvas as tarefas adiconadas.




